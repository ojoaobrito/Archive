\chapter{Introdução}
\label{chap:intro}

\section{Enquadramento}
\label{sec:amb} % CADA SECÇÃO DEVE TER UM LABEL
% CADA FIGURA DEVE TER UM LABEL
% CADA TABELA DEVE TER UM LABEL

O presente documento e aplicações computacionais associadas enquadram-se na Unidade Curricular de Projeto, inserida no 3º ano da Licenciatura em Engenharia Informática da Universidade da Beira Interior, no ano letivo 2018/2019.

\section{Motivação}
\label{sec:mot}
A interação homem/máquina tem-se tornado uma constante no panorama mundial. Assim, a capacidade de um computador equipado com uma câmara (ou outro dispositivo equiparado) reconhecer objetos ou pessoas que estejam à sua frente e tomar ações em conformidade, é uma caraterística cada vez mais apetecível. Inspirada pela naturaza humana, a área de Visão Computacional está cada vez mais presente nas sociedades atuais.

\section{Objetivos}
\label{sec:obj}
Este trabalho propõem-se a usar métodos de classificação de imagens, nomeadamente uma \ac{CNN}, para resolver dois problemas: 

\begin{enumerate}
    \item Validar um módulo de \textit{tracking} ao discriminar sequências de indivíduos diferentes de sequências de um mesmo indivíduo.
    \item Fazer o reconhecimento facial de um indivíduo face a um \emph{dataset} conhecido.
\end{enumerate}

\section{Organização do Documento}
\label{sec:organ}
% !POR EXEMPLO!
De modo a refletir o trabalho que foi feito, este documento encontra-se estruturado da seguinte forma:
\begin{enumerate}
\item O primeiro capítulo -- \textbf{Introdução} -- apresenta o enquadramento, motivação e objetivos do projeto, bem como a organização do presente documento.
\item O segundo capítulo -- \textbf{Redes Neuronais Convolucionais} -- descreve as bases teóricas de uma \ac{CNN}, passando também pelo papel do Córtex Visual Primário no reconhecimento de imagens por parte do ser humano.
\item O terceiro capítulo -- \textbf{Problema 1: Validação de Módulo de Tracking} -- apresenta o primeiro grande objetivo deste projeto, o método explorado na sua concretização e os resultados obtidos.
\item O quarto capítulo -- \textbf{Problema 2: Identificação de Indivíduos} -- apresenta, numa estrutura semelhante ao terceiro capítulo, o segundo objetivo que se procurou atingir.
\item O quinto capítulo -- \textbf{Conclusão e Trabalho Futuro} -- resume os conhecimentos e aptidões obtidos através da realização deste trabalho, realçando, ainda, aspetos que poderiam ser melhorados e/ou acrescentados.
\end{enumerate}