\chapter{Conclusões e Trabalho Futuro}
\label{chap:conc-trab-futuro}

\section{Conclusões Principais}
\label{sec:conc-princ}

Este trabalho serviu, acima de tudo, como uma introdução à temática das redes neuronais e respetiva implementação, para procurar dar resposta a alguns desafios. O conhecimento adquirido é e será mais valioso que qualquer resultado prático, formando uma base que se espera expandir no futuro.

\section{Trabalho Futuro}
\label{sec:trab-futuro}

De forma a suplementar o trabalho desenvolvido, algumas melhorias/adições podem ser destacadas:

\begin{enumerate}
    \item \textbf{Obter imagens com menos ruído (problema 1):} O facto de numa dada imagem de um indivíduo aparecerem, ora outros indivíduos ora partes dos corpos destes, não ajuda à avaliação das sequências. Muitas delas foram dadas como erradas, sendo que se tratavam da mesma pessoa (efeito indesejável do ruído existente).
    \item \textbf{Usar sequências com mais imagens (problema 2):} tendo observado as diferenças entre usar duas ou três imagens por sequência, o uso de mais imagens/combinações diferentes pode produzir resultados interessantes.
    \item \textbf{Alterar parâmetros/configurações na rede:} tal como descrito no capítulo \ref{chap3:sec:final}, certos parâmetros podem ter impacto no processo de aprendizagem de uma CNN e na performance sobre o conjunto de teste. Um modelo melhor contribui para um sistema de comparação de indivíduos melhor.
\end{enumerate}{}