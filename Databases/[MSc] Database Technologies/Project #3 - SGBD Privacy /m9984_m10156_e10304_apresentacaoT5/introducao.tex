%Indentação
\setlength{\parskip}{1em}
\setlength{\parindent}{0pt}

\chapter{Introdução}
\label{chap:intro}

\section{Enquadramento}
\label{sec:amb}

Este projeto foi realizado no contexto da unidade curricular de Tecnologias de Base de Dados, enquadrada no primeiro ano de Mestrado em Engenharia Informática da Universidade da Beira Interior, no ano letivo 2019/2020.

\section{Motivação}
\label{sec:mot}

O projeto foi proposto e será avaliado pelo professor docente Rui Cardoso. Visa a exploração de aspetos legais, éticos e profissionais na gestão de dados. Estes princípios serão, ainda, discutidos no âmbito de um \ac{SGBD}.

\section{Objetivos}
\label{sec:obj}
O presente projeto procurará expor a temática da gestão de dados em várias vertentes. Do ponto de vista profissional, podem ser considerados aspetos puramente técnicos (i.e. como são, efetivamente, guardados os dados). A título legal, será procurado o auxílio da legislação em vigor. Por fim, o caráter ético na gestão de informação terá destaque neste documento teórico, nunca esquecendo a importância que tem (ou deve ter) na prática.


\section{Constituição do grupo}
\label{sec:const_grupo}
A constituição do grupo de trabalho a que se deve a realização deste documento, e respetivo projeto, foi da responsabilidade dos próprios elementos, que se encontram aqui descritos:
\begin{itemize}
    \item João Brito, M9984;
    \item Luís Pereira, M10156;
    \item Carlos Esteves, E10304.
\end{itemize}

\section{Organização do Documento}
\label{sec:organ}
O presente documento encontra-se dividido em sete secções principais, nomeadamente:
\begin{enumerate}
    \item Introdução - é feita uma apresentação do projeto, identificando os objetivos e o seu âmbito;
    \item Gestão de dados: onde começam e acabam os limites? - são introduzidas questões e problemáticas associadas à gestão indevida de dados;
    \item Como dar garantias ao utilizador de que a gestão de dados é bem feita? - é abordada a necessidade de transparência de uma empresa ou entidade para com os seus utilizadores;
    \item Legislação existente - é feita uma revisão de algumas leis e entidades relevantes;
    \item Mecanismos de proteção de dados - são explorados aspetos mais técnicos no âmbito da gestão de dados num \ac{SGBD};
    \item Falhas e abusos na gestão de dados - são referidos alguns exemplos de uma gestão de dados incorreta;
    \item Conclusão e Trabalho Futuro - apresentam-se conclusões derivadas do trabalho desenvolvido, bem como algumas notas sobre possíveis melhorias ou tópicos passíveis de serem explorados.
\end{enumerate}