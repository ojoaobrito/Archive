\chapter{Falhas e abusos na gestão de dados}
\label{chap6:falhas}

\section{Introdução}
\label{chap6:intro}

Ao longo deste capítulo são abordados casos onde o respeito pela privacidade do utilizador foi negligenciado e, mais tarde, os respectivos dados acabaram por ser usados em práticas de exploração.




\section{\textit{Facebook–Cambridge Analytica}}
\label{chap6:exe1}

O escândalo \textit{Facebook–Cambridge Analytica} gerou controvérsia no inicio de 2018, quando foi revelado que a empresa \textit{Cambridge Analytica} tinha recolhido informação proveniente do \textit{Facebook} de milhares de utilizadores sem o consentimento destes. Os dados recolhidos foram mais tarde utilizados para propaganda política. O impacto causou uma queda no preço das acções do \textit{Facebook} e pedidos para uma regulamentação mais forte.

De uma forma geral, os dados foram recolhidos através de uma aplicação chamada \textit{"This Is Your Digital Life"}, criada por um dos trabalhadores das empresa. Mais tarde, centenas de milhares de utilizadores da rede preencheram um questionário com fins meramente académicos. No entanto, o design da rede permitiu não só a recolha da informação do questionário, mas também informação pessoal de todos os utilizadores da rede. Desta forma, informação dos utilizadores da redes foi prosperada pela companhia.

A informação prosperada foi, mais tarde, utilizada na campanha do concorrente Ted Cruz para a casa branca em 2015/2016 e no mercado publicitário para alterar a tendência de voto.




\section{\textit{Uber Tracking}}
\label{chap6:exe2}

Em 2014, a \textit{Uber} era uma das companhias com maior taxa de crescimento. Nesse mesmo ano, um dos funcionários da companhia usou recursos da empresa para detectar o local de um jornalista, que se encontrava atrasado para uma entrevista na empresa. A ferramenta utilizada ia contra a politica privada da companhia, em especifico, funcionários da companhia estavam proibidos de visualizar o histórico das viagens dos clientes, excepto em "casos legítimos".


\section{Conclusão}
\label{chap6:conc}

Os exemplos descritos acima, demonstram casos de mau uso de dados. O mau uso pode ser proveniente de várias fontes, desde funcionários, má gestão, entre outros. Porém, este mau uso resulta sempre na violação dos direitos dos utilizadores e, a longo prazo, estes dados acabam por ser usados de forma exploratória.