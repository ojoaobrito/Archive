\chapter{Legislação existente}
\label{chap4:legislacao}

\section{Introdução}
\label{chap4:intro}

Com o crescimento do sector tecnológico, apareceram problemas éticos e legais referentes à recolha, tratamento e uso dos dados de clientes. O mau uso ou o abuso dos dados dos clientes levaram a uma forte necessidade de regular a politica de dados. Como tal, sistemas de gestão de base de dados não fugiram à regra, isto é, como armazém principal de dados, este necessita uma forte regulação.

Ao longo deste capítulo são introduzidas entidades reguladoras e leis existentes criadas para defender a privacidade dos cidadãos e proibir o mau uso de informação.

\section{Comissão Nacional de Protecção de Dados}
\label{chap4:CNPD}

A \ac{CNPD} é uma entidade administrativa independente com poderes de autoridade. Esta funciona em cooperação com a Assembleia da Republica.

A sua finalidade consiste em controlar e fiscalizar o processamento de dados pessoais, de forma a conferir que estes estão de acordo com os direitos do homem e as liberdades e garantias consagradas na Constituição e na lei.

Dentro da legislação aprovada existem várias subclasses de leis: protecção de dados pessoais, saúde, comunicações electrónicas, videovigilância, trabalho, cartão de cidadão e cibercrime.

Dentro das várias leis, tome-se por exemplo o caso da protecção de dados pessoais. Aqui existem leis que definem como dados pessoais podem ou não ser usados para efeitos de prevenção, detecção, investigação ou qualquer outra tarefa que possa contribuir para a identificação de infracções penais, sujeitas a sanções. De modo a ilustrar o ponto, atente-se no seguinte artigo "Direito de acesso do titular dos dados aos seus dados pessoais" (Artigo 15º, Lei 59/2019). Isto é, um dado utilizador tem direito a perguntar se os seus dados estão a ser efectivamente utilizados para algum fim. Caso a resposta seja positiva, este tem ainda direito de requerer a remoção destes, as finalidades e fundamentos jurídicos, entre outros.

\section{Regulamento Geral sobre a Protecção de Dados}
\label{chap4:RGPD}

O \ac{RGPD} é um regulamento do direito europeu sobre privacidade e protecção de dados pessoais aplicável a todo o cidadão na União Europeia e Espaço Económico Europeu. Este foi criado em 2018 e tem como objectivo dar poder aos cidadãos europeus, bem como, métodos para controlar os seus dados pessoais. Adicionalmente, também é regulada a exportação de dados pessoais para fora da União Europeia.

O regulamento procura definir a maneira como são tratadas informações pessoais na União Europeia e é aplicável a toda e qualquer empresa que opera dentro do Espaço Económico Europeu, independentemente do seu país de origem.

De uma forma geral, todo o processo empresarial dentro do Espaço Económico Europeu é obrigado a ser desenhado de raiz e por um conjunto de medidas que respeitem o principio da protecção de dados desde o início da sua conceção, ou seja,  os dados usados devem ser guardados segundo métodos de anonimato de forma a evitar que estes sejam usados sem consentimento explícito e que seja impossível identificar o autor, sem informação adicional armazenada em paralelo.

Segundo o \ac{RGPD}, não é permitido o uso de dados fora do contexto legal especificado no regulamento, excepto quando o proprietário dos dados tenha dado o seu consentimento explicito a quem controla os dados. Note-se, que esta permissão pode ser revogada em qualquer instância pelo proprietário.

O responsável pela recolha de dados tem obrigatoriamente que declarar qualquer recolha de dados, declarar qual o enquadramento jurídico, a finalidade da recolha, qual o prazo durante o qual os dados ficam armazenados, e ainda, se estes serão partilhados para fora da União Europeia. Os utilizadores sujeito à recolha têm o direito de exigir a cópia dos dados recolhidos e exigir a remoção destes em determinadas circunstâncias.

\section{Conclusão}
\label{chap4:conc}

De uma forma geral, ao longo dos últimos anos foram criadas legislações e comissões que visam a protecção de dados pessoais. Por norma, empresas que façam recolha de dados estão sujeito a um conjunto de normas, que implica um total respeito pela privacidade do utilizador. Ao longo do presente documento será ainda possível tomar conhecimento de alguns exemplos nos quais não existiu respeito pela privacidade pessoal.